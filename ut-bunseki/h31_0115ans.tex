\documentclass[twocolumn]{article}
\usepackage[utf8]{inputenc}
\usepackage{indentfirst}
\usepackage{amsmath}
\usepackage{bm}
\usepackage{tikz}

\title{総合分析情報学 解答}
\author{ぴかちゅう}
\date{\today}

\begin{document}

\maketitle

\section*{Q1}
\subsection*{(1)}
行列\(A\)の固有値と固有ベクトルを求めます.ただし\(0 < \theta < \pi\)
\[
    A = \begin{pmatrix}
        \cos \theta & 0 & \sin 0      \\
        -1          & 1 & -1          \\
        \sin \theta & 0 & \cos \theta
    \end{pmatrix}
\]

固有値を求めるアルゴリズムは次の通りです\\

\begin{enumerate}
    \item 行列$A$の固有方程式を立てます
    \item 方程式を解いて固有値を求めます
    \item 固有値を固有ベクトルを求める式に代入します
\end{enumerate}

実際にやってみます\\

まずは固有方程式を立てます.固有値を$\lambda$とします.そうすると得られる固有方程式は次のようになります

$$
    |\lambda E - A| = 0\\
$$

これに単位行列$E$と与えられた行列$A$を代入すると

$$
    \left(
    \begin{array}{ccc}
            \lambda - \cos \theta & 0           & \sin \theta           \\
            1                     & \lambda - 1 & 1                     \\
            - \sin \theta         & 0           & \lambda - \cos \theta
        \end{array}
    \right) = 0
$$

この行列式を計算すると

$$
    (\lambda - \cos \theta)^2(\lambda - 1) = 0
$$

よって$\lambda = 1, \cos \theta$となります\\

次に固有ベクトルを求めます.固有ベクトルは

$$
(\lambda E - A)\bm{x} = 0
$$

に固有値を代入して$\bm{x}$について解くことで求まります

$\lambda = \cos \theta$を代入すると

\subsection*{(2)}
\subsection*{(3)}
\subsubsection*{1}
\end{document}
